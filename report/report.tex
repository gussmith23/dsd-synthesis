\documentclass{article}

\usepackage{fullpage}
\usepackage{parskip}
\usepackage{color}

\begin{document}
\title{Program Synthesis for DNA Strand Displacement}
\author{Kendall Stewart, Anthony Ha, Martin Kellogg}

\maketitle

\section{Introduction}

DNA computing seeks to leverage the chemical properties
of DNA molecules as a computational substrate.
From a medical perspective, DNA computing can enable the
construction of complex devices at a cellular scale for more precise
diagnostic tests or drug delivery. From a computer architecture perspective,
DNA is capable of performing massively parallel computations with very low
power consumption.

The feasibility of DNA computing was first demonstrated by encoding the
Hamiltonian Path Problem into DNA~\cite{adelman}. This encoding was constructed
by hand, but recent work has advanced DNA strand displacement reactions as a
primitive for DNA computation, because they can implement a wide
range of algorithms and devices, including boolean circuits~\cite{strands},
oscillators~\cite{dsd}, and neural networks~\cite{strandnn}.

Because of their boolean completeness, strand displacement reactions are
theoretically capable of implementing binary arithmetic operations. However,
some operations (such as multiplication) require a large number of
boolean gates, and strand-based boolean gates do not scale as well as
silicon-based gates; even moderately-sized systems begin to suffer noise
from unwanted interactions.

Because strand displacement reactions are capable of more than just boolean
logic, an arithmetic operation can be implemented with a set of
fewer, more complex strand displacement gates. However, reasoning about the
behavior of more complex reactions is difficult for humans. We hypothesize that
a program synthesis tool equipped with appropriate semantics for strand
displacement reactions could ``discover" gates that would be difficult for a
human to find.

A formal semantics for strand displacement reactions is provided in the
description of DSD~\cite{dsd}, a programming language for specifying and
simulating DNA strand displacement circuits. Given a set of initial strand
species (reactants), DSD applies reduction rules to infer the intermediate and
final strand species (products), and creates a directed reaction graph with
edges between reactants and products, weighted by the rate of each reaction.
This graph, along with ordinary differential equations derived from the rates,
are used to simulate the strand displacement system.

Ignoring rates, the structure of the unweighted reaction graph encodes
conditions that are necessary (but not sufficient) for some product to have a
particular meaning. For instance, if all paths leading to a product include both
initial species $A$ and initial species $B$, the product could
represent $A \land B$.

As a first pass at applying program synthesis to DNA strand
displacement, we have implemented an interpreter for the semantics of
DSD in Rosette~\cite{rosette}. Our tool enables us to construct
predicates that can reason about
the structure of the resulting graph (without reaction rates). Our goal
for the quarter was to determine whether
the solver can satisfy queries such as ``given
reactant $A$ and reactant $B$, find a system of gates where the
product represents $A \land B$" (where this meaning is derived from
the graph structure, as described above).

\textcolor{red}{TODO: Paragraph summarizing results.}

The primary contributions of our work are:
\begin{enumerate}
\item
  An implementation in Rosette of both the ''detailed''
  and ''default'' semantics for DSD, as an interpreter for
  strand displacement reactions.

\item
  An evaluation of this interpreter's usefulness for synthesizing
  implementations of simple boolean circuits as strand displacement
  reactions.

\item
  \textcolor{red}{A cool new gate or the like if we end up finding one.}
\end{enumerate}

\section{Overview of Our Approach}

\textcolor{red}{Describe how we took the semantics presented in the DSD paper and translated them into Rosette. A few paragraphs at most.}

\section{Algorithms and Encodings}

\textcolor{red}{Describe our use of transparent structs in Rosette,
  and the use of rule functions, etc. Maybe also mention the solver
  aided tests?}

\section{Results}

\textcolor{red}{How things actually turn out...}

\section{Teamwork}

To ensure that all three of us were involved in developing the encoding of
strand displacement reactions we used, much of the core infrastuctural code
that defines the encoding was developed by all three of us working on a single
screen. Once we had defined the core encoding, each team member worked on
making use of it in a particular way: Kendall, as the domain expert, worked
on implementing the rules and refining the encoding to perform better; Anthony
worked on building a implementing a normalization procedure (which required
some modifications to the encoding, as well) and a pretty printer/parser;
Martin created a test infrastructure (including a continuous integration server)
and then on writing verification procedures for the rules Kendall was
implementing (i.e. solver-aided exhaustive tests).

\section{Course Topics}

This project made heavy use of a solver-aided language: it is entirely dependent
on Rosette. In particular, we needed Rosette's ability to angelically execute
our symbolic programs (which represent reactions) and produce reactants that
cause the programs (reactions) to have the desired behavior---our synthesis
task. We also used bounded verification to test that our more complex
rules were correct. All topics we required were covered by the class.

\begin{thebibliography}{9}

\bibitem{adelman}
L. M. Adleman,
“Molecular computation of solutions to combinatorial problems,”
Science, vol. 266, no. 5187, pp. 1021–1024, Nov. 1994.

\bibitem{strands}
D. Y. Zhang and G. Seelig,
“Dynamic DNA nanotechnology using strand-displacement reactions,”
Nature Chemistry, vol. 3, no. 2, pp. 103–113, Feb. 2011.

\bibitem{dsd}
M. R. Lakin, S. Youssef, L. Cardelli, and A. Phillips,
“Abstractions for DNA circuit design,”
Journal of The Royal Society Interface, vol. 9, no. 68, Jul. 2011.

\bibitem{strandnn}
L. Qian, E. Winfree, and J. Bruck, “Neural network computation with DNA strand
displacement cascades,” Nature, vol. 475, no. 7356, pp. 368–372, Jul. 2011.

\bibitem{rosette}
  E. Torlak and R. Bodik, “A lightweight symbolic virtual machine for solver-aided host languages,” Programming Language Design and Implementation (PLDI), vol. 49, no. 6, pp. 530-541, 2014.

\end{thebibliography}

\end{document}
