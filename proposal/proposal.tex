\documentclass{article}

\usepackage{fullpage}
\usepackage{parskip}

\begin{document}
\title{Applying Program Synthesis to DNA Strand Displacement}
\author{Kendall Stewart, Martin Kellogg, Anthony Ha}

\maketitle

DNA Computing is a research area that seeks to leverage the chemical properties
of DNA molecules as a computational substrate. DNA computing is an appealing
idea for many reasons: from a medical perspective, it would enable the
construction of complex devices at a cellular scale, enabling more precise
diagnostic tests or drug delivery. From a computer architecture perspective,
DNA is capable of performing massively parallel computations with very low
power consumption.

The feasibility of DNA computing was first demonstrated by a DNA-based encoding
of the Hamiltonian Path Problem~\cite{adelman}. This encoding was constructed by
hand, but recent work has advanced DNA strand displacement
reactions~\cite{strands} as a primitive for DNA computation, in particular
because these reactions are capable of implementing boolean logic gates.

The boolean completeness of strand displacement reactions allows them to be
targeted by a compiler that knows how to translate arithmetic operations to
boolean circuits, and then to strand displacement reactions. This process would
result in a strand displacement system whose complexity is exponential in the
size of the problem.

Because strand displacement reactions are capable of more than just boolean
logic, an operation could potentially be implemented with a set of fewer, more
complex strand displacement gates, but reasoning about the behavior of more
complex reactions is difficult. We hypothesize that a program synthesis tool
equipped with appropriate semantics for strand displacement reactions could
``discover" gates that would be difficult for a human to find.

A formal semantics for strand displacement reactions is provided in the
description of DSD~\cite{dsd}, a programming language for specifying and
simulating DNA strand displacement circuits. Given a set of initial strand
species (reactants), DSD applies reduction rules to infer the intermediate and
final strand species (products), and creates a directed reaction graph with
edges between reactants and products, weighted by the rate of each reaction.
This graph, along with ordinary differential equations derived from the rates,
are used to simulate the strand displacement system.

Ignoring rates, the structure of the unweighted reaction graph encodes
conditions that are necessary (but not sufficient) for some product to have a
particular meaning. For instance, if all paths leading to a product include both
initial species $A$ and initial species $B$, the product could potentially
represent $A \land B$.

As a first pass at applying program synthesis to DNA strand displacement, we
propose to implement an interpreter for the semantics of DSD in Rosette, and
then construct predicates that can reason about the structure of the resulting
graph (without reaction rates). We will then see whether the solver can satisfy
queries such as ``given reactant $A$ and reactant $B$, find a system of gates
where the product represents $A \land B$" (where this meaning is derived from
the graph structure, as described above).

If this is successful, and we have additional time we could move on to more
complicated queries such as ``given reactants representing number $X$, and
reactants representing the number $Y$, find a system where the output represents
$X + Y$".  It is our hypothesis that the resulting systems may be smaller (in
the number of gates or number of reactions) than systems that are compiled from
known implementations of boolean circuits. This hypothesis could be tested by
encoding it as additional constraints for the synthesis problem.

To summarize, our proposed timeline of work for the rest of the term is as
follows:
\begin{enumerate}
\item
Write and test interpreter that implements DSD semantics in Rosette (2
weeks)

\item
Write and test predicates about graph structure (1 week)

\item
Synthesize simple systems implementing boolean circuits (1 week)

\item
Synthesize more complex systems (if time allows)

\item
Prepare final report and presentation (1 week)
\end{enumerate}

\begin{thebibliography}{9}

\bibitem{adelman}
L. M. Adleman,
“Molecular computation of solutions to combinatorial problems,”
Science, vol. 266, no. 5187, pp. 1021–1024, Nov. 1994.

\bibitem{strands}
D. Y. Zhang and G. Seelig,
“Dynamic DNA nanotechnology using strand-displacement reactions,”
Nature Chemistry, vol. 3, no. 2, pp. 103–113, Feb. 2011.

\bibitem{dsd}
M. R. Lakin, S. Youssef, L. Cardelli, and A. Phillips,
“Abstractions for DNA circuit design,”
Journal of The Royal Society Interface, vol. 9, no. 68, Jul. 2011.

\end{thebibliography}

\end{document}
